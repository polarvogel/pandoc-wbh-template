% pandoc WBH Prüfungsvorlage
%
% Diese Vorlage ist für Prüfungen an der Wilhelm-Büchner-Hochschule erstellt worden
% sie entspricht den Vorgaben für Hausarbeiten und Thesis zum aktuellen Zeitpunkt.
% 
% Autoren:
%
% Created:
% Changed: 26.06.2020
\PassOptionsToPackage{unicode$for(hyperrefoptions)$,$hyperrefoptions$$endfor$}{hyperref}

\documentclass[
$if(fontsize)$
	$fontsize$,
	$else$
	11pt,
$endif$
$if(papersize)$
	$papersize$paper,
	$else$
	a4paper,
$endif$
$if(lang)$
    $babel-lang$,
$endif$
    bibliography=totocnumbered,
    listof=totocnumbered,
    toc=flat,
    numbers=noenddot % Numbers without dots
$for(classoption)$
  	$classoption$$sep$,
$endfor$
]{scrartcl}

$if(lang)$
% Support different languages
% default: en
% -----------------------------------------------------------------------
\usepackage[shorthands=off,$for(babel-otherlangs)$$babel-otherlangs$,$endfor$main=$babel-lang$]{babel}

$if(babel-newcommands)$
$babel-newcommands$
$endif$

%\usepackage[utf8]{inputenc}
$endif$
\usepackage{textcomp}   % Erweitert den Zeichensatz
\usepackage[onehalfspacing]{setspace} % Der 1,5-zeilige Zeilenabstand

\usepackage[nottoc,numbib]{tocbibind}   % No the numbers befor lot and lof and add them to toc

\usepackage{fancyhdr}
\usepackage{tabularx}
\usepackage[usenames,dvipsnames]{xcolor}    % Colorize latex

% Leitfaden Vorgabe für Ränder:
% Links:            2cm oder mehr
% Oben und unten:   2cm
% Rechts:           4cm (Korrekturrand)
% ------------------------------------------------------------
$if(geometry)$
\usepackage[$if(geometry)$$for(geometry)$$geometry$$sep$,$endfor$$else$a4paper,margin=2cm,bottom=2.5cm,top=25mm,left=25mm,right=40mm,headsep=10,footskip=12mm$endif$]{geometry}
$endif$

\usepackage[right]{eurosym}
\usepackage[printonlyused]{acronym}
\usepackage{subfig}
\usepackage{floatflt}
\usepackage{colortbl}
\usepackage{paralist}
\usepackage{array}
\usepackage{parskip}
\usepackage[right]{eurosym}
\usepackage{helvet}
\usepackage{graphicx}
\usepackage[export]{adjustbox} % also loads graphicx, to have max width for graphics
\usepackage{pdfpages}
\usepackage{tikz}
\usepackage{amsmath}												% For pandoc extensive `amsmath` collection of symbols for typesetting ordinary math
\usepackage{amsfonts}												% More symboles for exotic currency notation and engeneering diagrams
\usepackage{amssymb}												% More symboles for exotic currency notation and engeneering diagrams
\usepackage{siunitx}                        % For using SI Units https://www.ctan.org/pkg/siunitx
$if(lang)$
\sisetup{locale = $lang/uppercase$}
$endif$
\usepackage[a-1b]{pdfx} % This needs to be befor hyperref package and is to generate PDF as PDF/A Standard
\usepackage{abstract} % For nice abstract formatting
% ----------------------------------------------------------------------------------------------------------
% Firstname + Lastname
% ----------------------------------------------------------------------------------------------------------

$if(student.firstname)$
% Firstname is set, assume name is Firstnaem + Lastname
\def \studentname{$student.firstname$ $student.lastname$}
$else$
% Firstname is not set, studentname = name
\def \studentname{$student.name$}
$endif$

% ----------------------------------------------------------------------------------------------------------
% Aufgabenstellung
% ----------------------------------------------------------------------------------------------------------
% Debug: 
% include: $assignment.include$ file: $assignment.file$ pages: $assignment.pages$

\def \assignment{
    
% Regular include for historic reasons
%------------------
% include assignment is on
% \includepdf[pages=$assignment.pages$]{$assignment.file$}
% New include for historic reasons
%------------------

% check if assignment.file is set at all 
$if(assignment.file)$
% check if form shall be filled out
$if(assignment.fillform)$
\includepdf[pagecommand={  % 23mm, 6mm
\begin{tikzpicture}[remember picture, overlay, x=1mm,y=1mm,%
mybox/.style={rectangle,minimum width=56mm, draw opacity=0.0, line width=0,  minimum height=8mm, align=left,text width=56mm},%
info/.style={mybox,draw=black,align=left}]
\node at (34,-36) [info] {$student.firstname$};
\node at (95,-36) [info] {$student.lastname$};
\node at (34,-49) [info] {$student.strasse$};
\node at (95,-49) [info] {$student.ort$}; 
\node at (34,-61) [info] {$student.matrikelnr$};
\node at (95,-61) [info] {$studium.studiengangnr$};
\end{tikzpicture}}, pages=1]{$assignment.file$}
$if(assignment.multipage)$ % check if we need to include the rest of the document
\includepdf[pages=$assignment.pages$]{$assignment.file$} % Include rest of document
$endif$
$else$
\includepdf[pages=$assignment.pages$]{$assignment.file$}
$endif$
$else$
%include assignment was off
$endif$
}


$if(tables)$
\usepackage{longtable,booktabs} 			% This two Packages are needet for Pandoc Table support. Issue is opened: https://github.com/jgm/pandoc/issues/1023
$endif$

\usepackage{multirow}

% Solution - for double underlined solutions in formulas/equations, uses \underline twice for double underline 
\newcommand{\solution}[1]{\underline{\underline{#1}}}


$if(graphics)$
\makeatletter
\def\maxwidth{\ifdim\Gin@nat@width>\linewidth\linewidth\else\Gin@nat@width\fi}
\def\maxheight{\ifdim\Gin@nat@height>\textheight\textheight\else\Gin@nat@height\fi}
\makeatother
% Scale images if necessary, so that they will not overflow the page
% margins by default, and it is still possible to overwrite the defaults
% using explicit options in \includegraphics[width, height, ...]{}
\setkeys{Gin}{width=\maxwidth,height=\maxheight,keepaspectratio}
% Set default figure placement to htbp
\makeatletter
\def\fps@figure{htbp}
\makeatother
$endif$

% Support hyperref with colorisation
% -------------------------------------------------------------------------
\usepackage{xcolor}
\IfFileExists{xurl.sty}{\usepackage{xurl}}{} % add URL line breaks if available
\IfFileExists{bookmark.sty}{\usepackage{bookmark}}{\usepackage[pdfpagelabels=true]{hyperref}}
\urlstyle{same} % disable monospaced font for URLs

$if(verbatim-in-note)$
	\VerbatimFootnotes % allow verbatim text in footnotes
$endif$

$if(links-as-notes)$
	% Make links footnotes instead of hotlinks:
	\DeclareRobustCommand{\href}[2]{#2\footnote{\url{#1}}}
$endif$

$if(strikeout)$
	\usepackage[normalem]{ulem}
	% Avoid problems with \sout in headers with hyperref
	\pdfstringdefDisableCommands{\renewcommand{\sout}{}}
$endif$

\setlength{\emergencystretch}{3em} % prevent overfull lines

% PDF Metadata
% ------------------------------------------------------------------
\hypersetup{
	unicode=false, 
	pdftoolbar=true, 
	pdfmenubar=true, 
	pdffitwindow=false, 
	pdfstartview={FitH},
	pdftitle={$if(title)$$title$$endif$$if(aufgabe.code)$: $aufgabe.code$$endif$},
	pdfauthor={$if(author)$$author$$else$\studentname$endif$$if(student.matrikelnr)$, Matrikelnummer: $student.matrikelnr$$endif$},
$if(studium.studiengang)$
	pdfsubject={Studiengang: $studium.studiengang$}, 
$endif$
	pdfcreator={\LaTeX\ with package \flqq hyperref\frqq via pandoc},
	pdfproducer={pdfTeX \the\pdftexversion.\pdftexrevision},
	pdfkeywords={$if(aufgabe.typ)$$aufgabe.typ$,$endif$ $if(student.matrikelnr)$$student.matrikelnr$,$endif$ $if(aufgabe.code)$$aufgabe.code$,$endif$ $if(keywords)$ $for(keywords)$ $keywords[, ]$ $endfor$$endif$},
	pdfnewwindow=true,
$if(lang)$
	pdflang=$lang$,
$endif$
	pdfdisplaydoctitle=true, 
$if(colorlinks)$
	colorlinks=true,
	linkcolor=$if(linkcolor)$$linkcolor$$else$gray$endif$,
	filecolor=$if(filecolor)$$filecolor$$else$magenta$endif$,
	citecolor=$if(citecolor)$$citecolor$$else$gray$endif$,
	urlcolor=$if(urlcolor)$$urlcolor$$else$black$endif$,
$else$
	hidelinks,
$endif$
}

\usepackage{filecontents}
\begin{filecontents*}{\jobname.xmpdata}
\Keywords{$if(aufgabe.typ)$$aufgabe.typ$\sep $endif$ $if(student.matrikelnr)$$student.matrikelnr$\sep $endif$ $if(aufgabe.code)$$aufgabe.code$\sep $endif$ $if(keywords)$ $for(keywords)$ $keywords[\sep ]$ $endfor$$endif$} 
\Title{$if(title)$$title$$endif$$if(aufgabe.code)$: $aufgabe.code$$endif$} 
$if(abstract)$\Subject{$abstract$} $endif$
\Author{$if(student.firstname)$$student.firstname$ $student.lastname$$else$$student.name$$endif$} 
$if(hochschule.name)$\Org{$hochschule.name$}$endif$
$if(doi)$\Doi{$doi$}$endif$
\Language{de-De, en-En}
%\Publisher{}
\end{filecontents*}

% Designing blockquote
% ------------------------------------------------------------------
\definecolor{blockquote-border}{RGB}{221,221,221}
\definecolor{blockquote-text}{RGB}{119,119,119}
\usepackage{mdframed}
\newmdenv[rightline=false,bottomline=false,topline=false,linewidth=3pt,linecolor=blockquote-border,skipabove=\parskip]{customblockquote}
\renewenvironment{quote}{\begin{customblockquote}\list{}{\rightmargin=0em\leftmargin=0em}%
\item\relax\color{blockquote-text}\ignorespaces}{\unskip\unskip\endlist\end{customblockquote}}

% Syntax Highlighting with colors
% -----------------------------------------------------------------
$if(highlighting-macros)$
$highlighting-macros$
$endif$
$if(verbatim-in-note)$
\usepackage{fancyvrb}
$endif$

$if(listings)$
  \usepackage{listings}
  \newcommand{\passthrough}[1]{#1}
  \lstset{basicstyle=\footnotesize, captionpos=b, breaklines=true, showstringspaces=false, tabsize=2, frame=lines, numbers=left, numberstyle=\tiny, xleftmargin=2em, framexleftmargin=2em}
	\makeatletter
	\def\l@lstlisting#1#2{\@dottedtocline{1}{0em}{1em}{\hspace{1,5em} Lst. #1}{#2}}
	\makeatother
$endif$

\renewcommand{\familydefault}{\sfdefault}

% Pandoc tightlisting
% ------------------------------------------------------------------
\providecommand{\tightlist}{%
  \setlength{\itemsep}{0pt}\setlength{\parskip}{0pt}}

% Support for citation
% -------------------------------------------------------------------
$if(natbib)$
\usepackage[$natbiboptions$]{natbib}
\bibliographystyle{$if(biblio-style)$$biblio-style$$else$plainnat$endif$}
$endif$
$if(biblatex)$
\usepackage[backend=bibtex]{biblatex}
	$for(bibliography)$
	\addbibresource{$bibliography$}
	$endfor$
$endif$

\usepackage[T1]{fontenc}

$if(csl-refs)$
\newlength{\cslhangindent}
\setlength{\cslhangindent}{1.5em}
\newlength{\csllabelwidth}
\setlength{\csllabelwidth}{3em}
\newlength{\cslentryspacingunit} % times entry-spacing
\setlength{\cslentryspacingunit}{\parskip}
\newenvironment{CSLReferences}[2] % #1 hanging-ident, #2 entry spacing
 {% don't indent paragraphs
  \setlength{\parindent}{0pt}
  % turn on hanging indent if param 1 is 1
  \ifodd #1
  \let\oldpar\par
  \def\par{\hangindent=\cslhangindent\oldpar}
  \fi
  % set entry spacing
  %\setlength{\parskip}{#2\cslentryspacingunit}
 }%
 {\par}
\usepackage{calc}
\newcommand{\CSLBlock}[1]{#1\hfill\break}
\newcommand{\CSLLeftMargin}[1]{\parbox[t]{\csllabelwidth}{#1}}
\newcommand{\CSLRightInline}[1]{\parbox[t]{\linewidth - \csllabelwidth}{#1}\break}
\newcommand{\CSLIndent}[1]{\hspace{\cslhangindent}#1}
$endif$

%$if(csl-refs)$
%\newlength{\cslhangindent}
%\setlength{\cslhangindent}{1.5em}
%\newenvironment{CSLReferences}%
%  {$if(csl-hanging-indent)$\setlength{\parindent}{0pt}%
%  \everypar{\setlength{\hangindent}{\cslhangindent}}\ignorespaces$endif$}%
%  {\par}
%$endif$

$if(numbersections)$
	\setcounter{secnumdepth}{$if(secnumdepth)$$secnumdepth$$else$5$endif$}
	$else$
	\setcounter{secnumdepth}{$if(secnumdepth)$$secnumdepth$$else$5$endif$}
	%\setcounter{secnumdepth}{-\maxdimen} % remove section numbering
$endif$
$if(acronym)$
% This is for adding Acronym with SI units correct
\newcommand{\siacl}[3]{\num{#1} \acl{#2} (\si{#3})}
$endif$
$if(hyphenation)$
\hyphenation{$for(hyphenation)$$hyphenation$$sep$ $endfor$}
$endif$


% --------------------------------------------------------------
% Include pandoc autogenerated headers
% --------------------------------------------------------------
$for(header-includes)$
$header-includes$
$endfor$

% --------------------------------------------------------------
% begin document
% --------------------------------------------------------------
\begin{document}


%
% Assignment
% -------------------
$if(assignment.file)$
$if(assignment.beforetitle)$
\assignment
$endif$
$endif$

% -------------------------------------------------------------
% Kopf und Fußzeile
% -------------------------------------------------------------
\renewcommand{\sectionmark}[1]{\markright{#1}}
\renewcommand{\leftmark}{\rightmark}
\pagestyle{fancy}
\lhead{}
\chead{}
\rhead{\thesection\space\contentsname}
\lfoot{\tiny $if(aufgabe.typ)$$aufgabe.typ$ des Studenten: $endif$$if(author)$$author$$else$\studentname$endif$\,$if(student.matrikelnr)$(Matrikelnr.: $student.matrikelnr$)$endif$ $if(studium.studiengang)$Studiengang: $studium.studiengang$$endif$ $if(aufgabe.code)$- Prüfung: $aufgabe.code$ $endif$}
\cfoot{}
\rfoot{\ \linebreak Seite \thepage}
\renewcommand{\headrulewidth}{0.4pt}
\renewcommand{\footrulewidth}{0.4pt}

% Vorspann
\renewcommand{\thesection}{\Roman{section}}
\renewcommand{\theHsection}{\Roman{section}}
\pagenumbering{Roman}

% Pagebreak after each Section
\let\oldsection\section
\renewcommand\section{\clearpage\oldsection}

% -------------------------------------------------------------
% Titelseite
% -------------------------------------------------------------
\begin{titlepage}
% Ränder für Titleseite werden neugesetzt
% Links:            2cm 
% Oben und unten:   2cm
% Rechts:           2cm 
\newgeometry{$if(titlepage)$$for(titlepage)$$titlepage$$sep$,$endfor$$else$top=20mm,left=20mm,right=20mm,bottom=20mm$endif$}
\begin{center}
$if(logo)$ 
    \includegraphics[max width=\textwidth]{$logo$}\\
$else$
    $-- Do Nothing if no logo is defined --$
$endif$ 
$if(hochschule.name)$ 
$if(logo)$
    $-- Do Nothing if the logo is defined --$
$else$   
    \Huge \textbf{$hochschule.name$} \\
$endif$   
$endif$

    \vspace*{1cm}
	
$if(aufgabe.typ)$
	\Huge
    \textbf{$aufgabe.typ$}\\
    \vspace*{0.3cm}
$endif$
$if(studium.fachbereich)$
	\large
    \textbf{Fachbereich $studium.fachbereich$}\\
		\vspace*{1cm}
$else$
$if(studium.studiengang)$
	\large
    \textbf{Studiengang: $studium.studiengang$}\\
		\vspace*{1cm}
$endif$
$endif$
$if(title)$
	\Huge
    \textbf{$title$} \\
$endif$
	\vspace*{0.3cm}
	\large
$if(aufgabe.code)$
    $aufgabe.code$ \\ % Aufgabencode
$endif$
	\vspace*{0.5cm}
$if(studium.fach)$
    \textbf{$studium.fach$}\\
$endif$
	\vspace*{0.5cm}
	
	\vfill
	\normalsize
	\newcolumntype{x}[1]{>{\raggedleft\arraybackslash\hspace{0pt}}p{#1}}
	\begin{tabular}{x{6cm}p{7.5cm}}
$if(studium.fachbereich)$
$if(studium.studiengang)$
        \rule{0mm}{4ex}\textbf{Studiengang:} &      $studium.studiengang$ \\
$endif$
$endif$
$if(student)$
        \rule{0mm}{4ex}\textbf{Student:} &          \studentname
$if(student.email)$
                                            \newline $student.email$
$endif$ \\
$endif$
$if(student.matrikelnr)$
        \rule{0mm}{4ex}\textbf{Matrikelnummer:} &   $student.matrikelnr$ \\
$endif$
$if(betreuer)$
        \rule{0mm}{4ex}\textbf{Betreut durch:} &    $for(betreuer)$$betreuer$$sep$\newline $endfor$\\
$endif$
$if(date)$
        \rule{0mm}{4ex}\textbf{Abgabedatum:} &      $date$ \\ % Fixed date was set
$else$
        \rule{0mm}{4ex}\textbf{Abgabedatum:} &      \today \\ % No date set, use today
$endif$
	\end{tabular}
	\vfill
$if(hochschule.adresse)$
    $if(hochschule.name)$$hochschule.name$, $endif$$hochschule.adresse$
$endif$
\end{center}
\end{titlepage}
\restoregeometry    % Zurück auf Standard Page Geometry
\pagebreak

% -------------------------------------------------------------
% Assignment
% -------------------------------------------------------------

$if(assignment.file)$
$if(assignment.beforetitle)$
$-- Do Nothing if the assignment is before the title! --$
$else$
\assignment
$endif$
$endif$


$if(skipfirstpage)$ % Skip first page count, if skipfirstpage = true
\clearpage
\setcounter{page}{1}
$endif$

$if(abstract)$
% --------------------------------------------------------------
% Abstract
% --------------------------------------------------------------
\begin{abstract}
	$abstract$
\end{abstract}
$endif$


$for(include-before)$
% --------------------------------------------------------------
% include-before Section
% --------------------------------------------------------------
$include-before$
$endfor$

$if(toc)$
$if(toc-title)$
\renewcommand*\contentsname{$toc-title$}
$endif$
{
$if(colorlinks)$
\hypersetup{linkcolor=$if(toc-color)$$toc-color$$else$black$endif$}		% Setup the link color of the toc
$endif$

\setcounter{tocdepth}{$toc-depth$}

% --------------------------------------------------------------
% Inhaltsverzeichnis
% --------------------------------------------------------------

\singlespacing
\rhead{$if(lang)$INHALTSVERZEICHNIS$else$TABLE OF CONTENTS$endif$} % Set Headline right to INHALTSVERZEICHNIS
\renewcommand{\contentsname}{I $if(lang)$Inhaltsverzeichnis$else$Table of contents$endif$}
\phantomsection % Needet to hyperref the section
%\addcontentsline{toc}{section}{\texorpdfstring{I \hspace{0.35em}$if(lang)$Inhaltsverzeichnis$else$Table of contents$endif$}{$if(lang)$Inhaltsverzeichnis$else$Table of contents$endif$}}
\addtocounter{section}{1} % reset the section counter to 1
\tableofcontents
\pagebreak
}
$endif$

% --------------------------------------------------------------
% Einrichtung der Kopfzeile
% --------------------------------------------------------------
\renewcommand{\sectionmark}[1]{\markright{#1}}
\renewcommand{\subsectionmark}[1]{}
\renewcommand{\subsubsectionmark}[1]{}
\lhead{\thesection\space\nouppercase{\rightmark}}
\rhead{} %hier kann die rechte Seite der Kopfzeile editiert werden!

\onehalfspacing
\renewcommand{\thesection}{\arabic{section}}
\renewcommand{\theHsection}{\arabic{section}}
\setcounter{section}{0}
\pagenumbering{arabic}
\setcounter{page}{1}
%\renewcommand{\includegraphics}[1][]{\includegraphics[width=0.9\columnwidth,keepaspectratio]{#1}}


% --------------------------------------------------------------
% Inhalt
% --------------------------------------------------------------

$body$

% --------------------------------------------------------------
% Verzeichnisse
% --------------------------------------------------------------
\pagebreak % wird benötigt um den header erst auf der folgeseite beginnen zu lassen.
\lhead{} % remove the left part of header


% Nachspann
\renewcommand{\thesection}{\Roman{section}}
\renewcommand{\theHsection}{\Roman{section}}
\pagenumbering{Roman}

% Pagebreak after each Section
%\let\oldsection\section
%\renewcommand\section{\clearpage\oldsection}

%\setcounter{section}{2}
%\setcounter{page}{2}

$if(acronym)$
% --------------------------------------------------------------
% Abkürzungsverzeichnis
% --------------------------------------------------------------
\rhead{$if(lang)$ABKÜRZUNGSVERZEICHNIS$else$LIST OF ABBREVIATIONS$endif$}
\section*{Abkürzungsverzeichnis}
\addcontentsline{toc}{section}{Abkürzungsverzeichnis}

\begin{acronym}[$acronym.longest$]                         % längste Abkürzung steht in eckigen Klammern
    %\setlength{\itemsep}{-\parsep}             % geringerer Zeilenabstand
$for(acronym.list)$
	\acro{$it.id$}[$it.short$]{$it.long$}
	$if(it.longplural)$
	\acroplural{$it.id$}[$if(it.shortplural)$$it.shortplural$$else$$it.short$$endif$]{$it.longplural$}
	$endif$
$endfor$
\end{acronym}
\pagebreak
$endif$

$if(lof)$
% --------------------------------------------------------------
% Abbildungsverzeichnis
% --------------------------------------------------------------
\rhead{$if(lang)$ABBILDUNGSVERZEICHNIS$else$LIST OF FIGURES$endif$}
\listoffigures
\pagebreak
$endif$

$if(lot)$
% -----------------------------------------------------------------
% Tabellenverzeichnis
% -----------------------------------------------------------------
\rhead{$if(lang)$TABELLENVERZEICHNIS$else$LIST OF TABLES$endif$}
\listoftables
\pagebreak
$endif$

% --------------------------------------------------------------
% Literaturverzeichnis
% --------------------------------------------------------------

$if(natbib)$
    $if(bibliography)$
        $if(biblio-title)$
            $if(has-chapters)$
\renewcommand\bibname{$biblio-title$}
            $else$
\renewcommand\refname{$biblio-title$}
            $endif$
        $endif$
        $if(beamer)$
\begin{frame}[allowframebreaks]{$biblio-title$}
  \bibliographytrue
        $endif$
  \bibliography{$for(bibliography)$$bibliography$$sep$, $endfor$}
        $if(beamer)$
\end{frame}
        $endif$
    $endif$
$endif$

$if(biblatex)$
    $if(beamer)$
\begin{frame}[allowframebreaks]{$biblio-title$}
  \bibliographytrue
  \printbibliography[heading=none]
\end{frame}
    $else$
\printbibliography$if(biblio-title)$[title=$biblio-title$]$endif$
    $endif$
$endif$

$for(include-after)$
$include-after$

$endfor$

$if(insurance)$
% -------------------------------------------------------------
% Erklärung zur Selbstständigkeit
% -------------------------------------------------------------
\newpage
\thispagestyle{empty}
\begin{center}
	\vspace*{5em}
	\huge\textbf{Eidesstattliche Erklärung}\\
\end{center}
\vspace{2em}

\studentname, $student.matrikelnr$

Hiermit erkläre ich an Eides statt, dass ich diese Arbeit selbstständig abgefasst und keine anderen als die angegebenen Quellen und Hilfsmittel benutzt habe. Sämtliche Stellen der Arbeit, die im Wortlaut oder dem Sinne nach Publikationen oder Vorträgen anderer Autoren entnommen sind, habe ich als solche kenntlich gemacht. Ich bin mit einer Plagiatsprüfung einverstanden.


Die Arbeit wurde bisher keiner anderen Prüfungsbehörde vorgelegt und auch noch nicht veröffentlicht.

\vspace{4em}
\begin{minipage}{\linewidth}
	\begin{tabular}{p{15em}p{15em}}
		Datum: &  .......................................................\\
		& \centering (\studentname)\\
	\end{tabular}
\end{minipage}
$endif$



$for(include-after)$
$include-after$

$endfor$
\end{document}
